\documentclass{article}

\usepackage[latin1]{inputenc}
\usepackage[english]{babel}
\usepackage[T1]{fontenc}
\usepackage{amsmath,amssymb}
\usepackage{fancyhdr}
\usepackage{hyperref}
\usepackage{graphicx}
\usepackage{tabularx}
\usepackage{float}


\fancyfoot[OC]{\textit{\thepage}}

\title{Samarbejdsaftale BNDN}
\date{\today}

\author{Asbj\o rn Fjelbro Steffensen (afjs@itu.dk)\\ Thomas Stoy Dragsb\ae k (thst@itu.dk)\\ Nicki Hjorth J\o rgensen (nhjo@itu.dk)\\ Jacob Cholewa (jbec@itu.dk)\\ Jakob Merrild (jmer@itu.dk)\\ Mathias Pedersen (mkin@itu.dk)}

\begin{document}

\maketitle
\newpage

\section{Kontaktoplysninger}
Thomas Dragsb\ae k, thst@itu.dk, 20553025 \\
Nicki Hjorth J\o rgensen, nhjo@itu.dk, 53343780 \\
Asbj\o rn Steffensen, afjs@itu.dk, 30354713 \\
Jacob Cholewa, jbec@itu.dk, 60853600 \\
Mathias Pedersen, mkin@itu.dk, 22640098 \\
Jakob Merrild, jmer@itu.dk, 28762742

\section{Gruppens m\o der}
Tirsdag kl. 09-17 p\aa \ ITU og eventuelt torsdage, hvor der ikke er F\# undervisning.

\section{Arbejdsform}
L\o st SCRUM med fokus p\aa \ rollerne (product owner, scrum master) og overblik over/uddelegering af tasks. \\ \\
Product owner: Jacob Cholewa \\
Scrum master: Mathias

\subsection{Definition of done}
\begin{itemize}
\item \textbf{Tekst}: Man har selv l\ae st teksten igennem og tjekket om korrektur samt indhold er tilfredsstillende. Derefter skal det reviewes og godkendes af gruppen i et review m\o de. S\aa fremt det ikke godkendes af gruppen, skal teksten omskrives med hensyntagen til gruppens feedback.
\item \textbf{Kode}: Velskrevet, dokumenteret, unit testet og committet kode, der er blevet reviewet af en anden i gruppen. Dokumentation indeb\ae rer summary, OCL (@pre, @post, @invariant i <remarks>) samt exceptions til klasser og metoder og inline kommentering, hvor det giver mening.
\end{itemize}

\section{Ambitionsniveau}
Vi har et h\o jt ambitionsniveau, idet vi alle er enige om at prioritere dette kursus h\o jt. Is\ae r vil der v\ae re meget fokus p\aa \ rapporten.

\section{Straf}
Et medlem kan smides ud af gruppen, hvis der har v\ae ret to kollektive advarsler uden den \o nskede effekt. Kollektive advarsler kan kun gives, hvis gruppens medlemmer minus 2 finder det n\o dvendigt. Eksempler p\aa \ ting, der kan udl\o se en advarsel: Ikke overholdte deadlines, manglende fremm\o de.

\end{document}