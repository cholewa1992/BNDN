\documentclass[../report.tex]{subfiles}
\begin{document}

\section{Business case}
It is difficult to break through as an artist as money is often a barrier since it is fairly expensive to release your work if you are not signed to a publisher. At the same time is difficult to get signed by a publisher since they will not sign a contract with an artist unless they are certain that there is a market for the art that the artist produces as a publisher's main focus is money. \textit{ArtShare} aims to accomodate this problem. With a publisher, the artist assigns the copyright of their art to the publisher, who then amongst other things collect royalties and distribute a share of these to the artist, license the art and promote the art to recording artists, film and television\footnote{``What is music publishing?'' - http://www.mpaonline.org.uk/FAQ}. The downside is that if you get a contract with a publisher they will take a big cut of your profits. This might result in work never being published.\\

\noindent ArtShare is a platform for artists to independently publish their work without having a huge portion of their profits devoured by the publisher. To maximise the artists' profit the platform will allow anyone to publish their own art and make it available for everyone to buy and enjoy. Whether it is movies, music, literature or any other form of digital art. Furthermore the platform should provide easy promotion of art which is popular among its users through its own promotion techniques as well as integration with various social media outlets. In other words, ArtShare will allow you to easily publish your art and give you all the positive things that comes with a publisher without taking a big cut of your profits. 

%\section{Business case}

%It is difficult to break through as an artist. We aim to accommodate this problem. Money is often a barrier as it is fairly expensive to release your work if you are not signed to a publisher. At the same time is difficult to get signed by a publisher since they will not sign a contract with an artist unless they are certain that there is a market for the art that the artist produces. Additionally if you get a contract with a publisher they will take a big cut of your profits. This often results in work never being published.\\

%\noindent We want to supply a platform for artists to independently publish their work without having a huge portion of their profits devoured by the publisher. To maximise the artists' profit the platform will allow anyone to publish their own art and make it available for everyone to buy and enjoy. Whether it is movies, music, literature or any other form of digital art.\\


%\noindent Furthermore the platform should provide easy promotion of art which is popular among its users through its own promotion techniques as well as integration with various social media outlets

%Our platform will facilitate a growth in the publication of art which is beneficial for cultural growth, but also benefit both artist and consumer by maximizing the artists profit and making the product cheaper for the consumer. 

\end{document}
