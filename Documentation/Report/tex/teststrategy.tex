\documentclass[../report.tex]{subfiles}

\def\mystrut(#1,#2){\vrule height #1 depth #2 width 0pt}

\newcolumntype{L}[1]{>{\mystrut(3ex,0ex)\let\newline\\\arraybackslash\hspace{0pt}}m{#1}}
\newcolumntype{C}[1]{>{\mystrut(3ex,0ex)\centering\let\newline\\\arraybackslash\hspace{0pt}}m{#1}}

\newcommand{\footnoteref}[1]{\textsuperscript{\ref{#1}}}

\begin{document}
\graphicspath{{img/}{../img/}}

\subsection{Test Strategy}

A very important part of every software project is to validate that the deliverable is working and fulfils its requirements. In this project the solution will therefore be tested to ensure a high quality and to prove that requirements have been meet.

Following tests strategies will be used to accomplish this.

\paragraph{Unit tests}
As per the developers definition of done; a code task to be done, working and unit-tested code must be committed to the project repository. In addition to this all code will be Quality assured by another developer. This will enforce the code quality and ensure that code in-fact perform the intended logic

\paragraph{Integration tests}

To ensure that the integration between subsystems are working as intended, a set of integration tests will be written. The integration tests will also be used under development. We're not using continues integration, but every integration test will be ran after changing a tested subsystem. This will insure that logic will not break due to changes and optimizations.

\paragraph{Usability tests}
As the non-functional requirements for this project include requirements for usability, a usability test will be executed to ensure that the requirements are met. A set of tasks will be written to cover these requirements and tested on a small target group matching the target group of the system on parameters of age, gender and IT knowledge.

\paragraph{Acceptance tests}
Acceptance tests are tests done to ensure that the system meets the system requirements. Acceptance testing is normally done by the customer, but as there is no customers to this project acceptance tests will be executed by the project group itself.

For every requirement a test case will be written. These test cases can either be tested though system testing, integration testing, unit tests or usability tests. System testing is a test case of the full working test where a task is manually carried out. 

\subsection{Reflection and results}
This section will contain the results and a brief reflection following from the test strategy.

\subsubsection{Results}
\paragraph{Unit test results}

233 unit tests have been written covering all classes containing logic. 221 of the tests are covering ShareIT and 12 covering ArtShare. All unit tests are compiling and is running without errors.

\paragraph{Integration test results}

A set of integration tests have been written to test that the Data Access Layer and the Business Logic Layer is working together as intended. This have been done by having a mock database. A set of test data have been written and and all methods in the Business layer have been tested to ensure the intended database operation. All 20 tests are compiling and is running without errors

Only integration tests have been implemented on these two subsystems as it was the only subsystems where logic relied on one another

\paragraph{Usability test results}


\paragraph{Acceptance tests}
For all requirements test cases have been written (See table \ref{testmatrix}) and every test have been ran. The result is shown in the test case matrix. Some of the failed tests have been corrected while others have been marked out of scope. Some requirements was taken out of scope and therefore the tests of course do not pass.

\paragraph{Reflection}

Through the test strategy a well tested and working solution has emerged. Every functional requirement, except for those out-scoped, have been proved through acceptance tests.

\newgeometry{left=1.5cm,top=1cm}

Explanation of abbreviations in the following table:
\begin{itemize}
\item UNT = Unittest
\item UST = Usability Test
\item AT = Acceptance Test
\item IT = Integration Test
\end{itemize}

\begin{longtable}{|L{2.1cm}|L{1cm}|L{12.4cm}|L{1.4cm}|}


\hline
\multicolumn{1}{|L{2.1cm}|}{ID} & \multicolumn{1}{L{1cm}|}{Type} & \multicolumn{1}{L{12.4cm}|}{Description} & \multicolumn{1}{L{1.4cm}|}{Status}  \\ \hline 
\endhead

FR-01-TC-01 & UST & Test if an end user is able to initiate a search within 2 minutes & Passed \cellcolor{green!60} \\ \hline
FR-02-TC-01& UNT& Test if a media item containing the search key in the title is returned in the MediaItemSearchResultDTO when searching for said search key& Passed \cellcolor{green!65} \\ \hline
FR-02-TC-02& UNT& Test if a media item containing the search key in the uploader name is returned in the MediaItemSearchResultDTO when searching for said search key& Failed\footnote{Out of scope} \cellcolor{red!65} \\ \hline
FR-02-TC-03&UST&Test if an end user is able to find a media item with a specific title within 2 minutes&Passed \cellcolor{green!60} \\ \hline
FR-03-TC-01&AT&Test if an end user is presented with a range of all media items matching the search key when searching &Passed \cellcolor{green!65} \\ \hline
FR-04-TC-01&UST&Test if an end user is able to find a media item with a specific title which is not on the first search result page within 2 minutes&Passed \cellcolor{green!60} \\ \hline
FR-05-TC-01&UST&Test if an end user is able to navigate to the details page of a media item in the search result within 2 minutes&Passed \cellcolor{green!60} \\ \hline
FR-05-TC-02&AT&By a given search ensure that all results have a link to its details page&Passed \cellcolor{green!65} \\ \hline
FR-06-TC-01&AT&Test if an end user is presented with a range of all media items matching the search key grouped by media type when searching &Passed \cellcolor{green!65} \\ \hline
FR-07-TC-01&UST&Test if an end user is able to create an Account within 5 minutes&Passed \cellcolor{green!60} \\ \hline
FR-08-TC-01&AT&In creating an account ensure that the user can enter a desired username and password&Passed \cellcolor{green!65} \\ \hline
FR-09-TC-01&UNT&Given a username, test if an exception is thrown if the username already exist in the database (case insensitive matching)&Passed \cellcolor{green!65} \\ \hline
FR-10-TC-01&AT&Test if an end user is presented with an informative message, if the username is unavailable&Passed \cellcolor{green!65} \\ \hline
FR-11-TC-01&UNT&Given a username that is not between 1 and 20 characters (both included), test if an exception is thrown&Passed \cellcolor{green!65} \\ \hline
FR-12-TC-01&UNT&Given a username that does not consist of only alphanumeric characters (a-zA-Z0-9), test if an exception is thrown&Passed \cellcolor{green!65} \\ \hline
FR-13-TC-01&UNT&Given a password that is not between 1 and 50 characters (both included), test if an exception is thrown&Passed \cellcolor{green!65} \\ \hline
FR-14-TC-01&UNT&Given a password that contains any whitespace character, test if an exception is thrown&Passed \cellcolor{green!65} \\ \hline
FR-15-TC-01&AT&Test if an end user is presented with informative messages when FR-09 - FR-14 are not met&Passed \cellcolor{green!65} \\ \hline

\captionsetup{belowskip=10pt}

\caption{\label{testmatrix} Test case matrix (full table in appendix)}

\end{longtable}

\end{document}