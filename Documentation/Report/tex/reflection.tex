\documentclass[../report.tex]{subfiles}

\begin{document}
\graphicspath{{img/}{../img/}}

%Proposed system
\section{Proposed system}
%\paragraph{Scope} 

\paragraph{Requirements} In defining the requirements of the system, the main focus has been to ensure that each requirement is well defined and unambigous. This has resulted in quite a few requirements, which have been very useful when implementing functionality. However, the potential of the thorough requirements was not fully utilized, since some developers decided to go with another solution at the time. Because of this some of the requirements in the test-matrix are failing, since the solution that is implemented in \textit{The System} is different from what was initially decided.

%grundige, ikke l�se dem

\paragraph{Mockups} In the process of defining all the functional requirements of ShareIt and ArtShare, it was decided to produce mockups of \textit{ArtShare}s GUI. These mockups would show all the functionality that \textit{ArtShare} was ideally thought to have, and thereby functional requirements could be derived from these mockups. Some of the created mockups can be seen in Appendix~\ref{app:mockups}. Furthermore the mockups helped the developers agree on the overall design of \textit{ArtShare}. The final version of \textit{ArtShare} reflects this by having the same functionality as depicted in the mockups.

%Design, Architecture & Implementation
\section{Design, Architecture \& Implementation}

\paragraph{Architecture \& Design}
The decision to use a layered architecture for \textit{ShareIt} was mainly influenced by a desire to separate the services from the storage.
However, this lead to a decision to make it easy to change the implementation of the Business Logic Layer (BLL) and Data Access Layer by use of the Abstract Factory Pattern and Bridge Pattern respectively.
The use of an Abstract Factory Pattern in the BLL makes it more difficult to implement new functionality.
This clashes with non-functional requirement \textbf{NFR-08} which specifies that it should be easy to implement new functionality.
When the decision was made to use the Abstract Factory Pattern it was however deemed more important, to be able to quickly create a dummy implementation of the BLL which could be deployed and used by the SMU team.
The Abstract Factory Pattern made it possible to easily switch this implementation with the actually implementation, when it was completed.

\paragraph{DTOs}
There are two points to reflect on concerning the DTOs with the system. Firstly, it would have been beneficial to use smaller DTOs with properties for the situations in which they would be used.
Secondly, \textit{ArtShare} has a lot of redundant code caused by the namespace problems when referencing the \textit{ShareIt} services.

\paragraph{Data model}
One can argue that the naming of entities in the data model is ambiguous and not the obvious choice. An example is the entity named "Entity", which advantageously could have been called "Media Item".

%Validation
\section{Validation}
	%NFR not tested? f.eks. 10 concurrent users
As a result of the well defined and thorough test strategy it can be guaranteed that the solution meets the functional requirements. However, some of the initial requirements have been out-scoped and are therefore not fulfilled. 

\paragraph{Usability tests}
Some of the functional requirements were covered by usability test cases. The received input from the these tests showed that four improvements on the design of \textit{ArtShare} could be made. Two of these inspired for a change of the download button to a different colour and a change in the positioning of the pagination buttons in the search result page. The last two were out-scoped, as there were no time for a big redesign. 

If all the improvements derived from the feedback had been implemented, the tests should have been rerun to ensure that the implementations actually improved the design of \textit{ArtShare}.

\section{SMU collaboration}
The experience of collaborating with the SMU team has been very interesting and educational, teaching us the importance of clear and precise communication.
It also helped showcase the differences in coding styles and practises, which are not as apparent when working only with ITU students since we are all taught the same practices. 
Despite of the issues along the way it is our impression that the SMU team was satisfied with the collaboration and so were we.


\end{document}