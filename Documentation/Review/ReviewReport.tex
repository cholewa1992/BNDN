\documentclass[]{article}
\usepackage{hyperref}

\title{Review of group 7's report}
\author{Asbj\o rn Fjelbro Steffensen (afjs@itu.dk)\\ Thomas Stoy Dragsb\ae k (thst@itu.dk)\\ Nicki Hjorth J\o rgensen (nhjo@itu.dk)\\ Jacob Benjamin Cholewa (jbec@itu.dk)\\ Jakob Merrild (jmer@itu.dk)\\ Mathias Kindsholm Pedersen (mkin@itu.dk)}

\begin{document}

\maketitle


\section*{What was reviewed}
Initial draft of the report that group 7\footnote{\url{https://wiki.smu.edu.sg/is411/2013T2-ITU-SMU\_Group\_wikis\#Team\_7}} is working on for the course Software Development in Large Teams with International Collaboration (BNDN). This included an introduction, sections on the proposed system and architecture, a test strategy and a section on the SMU collaboration.
\section*{What was approved}
\begin{itemize}
\item Nice todo list
\item Good structure of the report
\item Great that you are encrypting passwords
\item Good deployment diagram
\end{itemize}
\section*{What was not approved}
\begin{itemize}
\item \textbf{Overall}
	\begin{itemize}
	\item Missing a glossary
	\item Missing a business case
	\item Missing a conclusion
	\item Agree on terms to use. E.g. ``program''/``client'' and ``mediums/media'' \\
	\end{itemize}

\item \textbf{Introduction}
	\begin{itemize}
	\item The report is not a result of the collaboration, but a part of it
	\item Instead of refering to Netflix, try to explain what the service is
	\item Move the text from the footnote into the section
	\end{itemize}

\item \textbf{Proposed system}
	\begin{itemize}
	\item This section should not describe actual implementations. Instead it should give an overview of what the system is supposed to do.
	\item Consider including all of your requirements in the text
	\item Use ``must'' instead of ``should''
	\item It would make sense to split up some of the requirements, e.g. ``Administrator users should be able to add, update and delete media'' is basically three requirements.
	\item It would be a good idea to divide the non-functional requirements into a server section and a client section
	\item Some of the non-functional requirements are functional, e.g. ``The user should be able to access his account page from all views'' could be rewritten as ``Without prior knowledge of the client, a user must be able to navigate to his account page within 1 minute''
	\item Add ``inactivity'' to the glossary	
	\end{itemize}

\item \textbf{Proposed architecture}
	\begin{itemize}
	\item Argument for your choice of architecture (consider alternatives)
	\item Consider renaming the ``Data Storage Layer'' to ``Data Access Layer''
	\item Describe the different layers (include class diagrams)
	\item Write ER diagram instead of diagram
	\item The database design section is difficult to understand without looking in the appendix. Consider including a simple diagram without properties in the text
	\item It might be a good idea to argument for your choice of datamodel. It seems that the IsA relationship is unnecessary for Medium as none of the different types have unique properties.
	\item Use ``Data Access Layer'' instead of ``Database Communication Layer''
	\item The Facebook-authentication could be described better to avoid confusion
	\end{itemize}

\item \textbf{Test strategy}
	\begin{itemize}
	\item The overview does not give an overview of what to expect from this chapter. Consider moving the contents of this section to chapter 6
	\item In the section about roles and responsibilities you should just argument for your choices and save the reflection for later
	\item It is unclear whether section 4.3 is about back-ends or clients in different browsers
	\item The argument of using Nunit is vague as unit tests in a perfect world has only one assert
	\item Include ``not'' in the first line of section 4.5
	\item Regression testing does not only involve running all tests after fixing errors but also after adding new functionality.
	\item It would be a good idea to consider Integration Testing
	\item Consider using test cases to ensure that all requirements have been met
	\end{itemize}

\item \textbf{SMU collaboration}
	\begin{itemize}
	\item Section 6.1 should be split into an Overview section and an Issues section
	\item Consider dividing the Issues section into subsections, e.g. Communication, Culture, ...
	\item The paragraph at the bottom of page 21 is more of a summary/reflection than an issue
	\item There seems to have been a lot of communication issues. Consider writing a section on your agreements on the communication (who attends, how do we communicate, how often, ...)
	\end{itemize}
\end{itemize}
\end{document}
